\documentclass[10pt,a4paper]{article}
\usepackage[latin1]{inputenc}
\usepackage[spanish]{babel}
\usepackage[toc, page, header]{appendix}
\renewcommand{\appendixtocname}{Apendices}
\renewcommand{\appendixpagename}{Apendices}
\usepackage{amsmath}
\usepackage{amsfonts}
\usepackage{amssymb}
\usepackage{graphicx}
\usepackage{url}
\author{Santiago Videla}
\title{Especificaci\'on de Requerimientos de Software}
\pagestyle{headings}
\date{24 de septiembre de 2010}
\begin{document}
\maketitle
\pagebreak
\tableofcontents
\pagebreak

\section{Introducci\'on}
  \subsection{Prop\'osito}
  \paragraph{}
  El prop\'osito de este documento es la especificaci\'on de requerimientos de
software en el marco de la tesis de grado de la carrera Lic. en Cs. de la
Computaci\'on de la FaMAF - UNC, \textbf{``Dise\~no de vacunas atenuadas con
menor probabilidad de sufrir reversi\'on a la virulencia''}. Los requerimientos
del software son provistos por integrantes de FuDePAN en su car\'acter de
autores intelectuales de la soluci\'on que se pretende implementar y
colaboradores de dicha tesis.

  \paragraph{}
  A continuaci\'on se enumeran las personas involucradas en el desarrollo de la
tesis y que por lo tanto, representan la principal audiencia del presente
documento.
  \begin{itemize}
  \item Dra. Laura Alonso Alemany: Directora de tesis, FaMAF
  \item Daniel Gutson: Colaborador de tesis, FuDePAN
  \item Santiago Videla: Tesista, FaMAF
  \end{itemize}  
 
  \subsection{Alcance}
  El producto que se especifica en este documento se denomina \textbf{``vac-o''}
y su principal objetivo es encontrar secuencias gen\'omicas de pat\'ogenos que
mantengan funcionalidad como vacunas con una probabilidad m\'inima de sufrir
reversi\'on a la virulencia. El producto final debe proveer al usuario, la
capacidad de hacer uso del software mediante extensiones o \textit{plugins} que
implementen las caracter\'isticas particulares para cada pat\'ogeno. 
  
  La principal responsabilidad de vac-o es la de proveer a las extensiones, un
motor combinatorio de secuencias gen\'omicas utilizando diferentes estrategias
computacionales que conduzcan a la optimizaci\'on de los c\'alculos. Por otro
lado, las extensiones son responsables de evaluar las secuencias generadas
asignando un puntaje a cada una con el fin de que vac-o genere el listado de
secuencias resultante.

  En su versi\'on inicial, vac-o incluye una extensi\'on para la vacuna Sabin
contra la poliomielitis que podr\'ia ser tomada como gu\'ia para futuras
extensiones.

  \subsection{Descripci\'on general del documento}
  La estructura de este documento sigue las recomendaciones de la ``Gu\'ia para
la especificaci\'on de requerimientos de la IEEE'' (IEEE Std 830-1998).

  En la secci\'on \ref{section-desc-gral} se presenta una descripci\'on general
de vac-o, sus principales funcionalidades, interfaces y perfiles de usuarios.
  
  En la secci\'on \ref{section-req} se detallan los requerimientos funcionales
espec\'ificos de vac-o y los principales atributos que debe cumplir el software.

\section{Descripci\'on General}
  \label{section-desc-gral}
  \subsection{Perspectiva del Producto}   
    Al momento de la confecci\'on de este documento, no existen productos de
software que brinden las funcionalidades que se pretenden implementar. En este
sentido, vac-o representa una innovaci\'on en el \'area de la bioinform\'atica,
cuyo principal objetivo es la optimizaci\'on de vacunas basadas en virus
atenuados.
    
    Uno de los problemas de estas vacunas es su potencialidad de revertir a la
virulencia. En sucesivas replicaciones, los virus atenuados pueden acumular
mutaciones que les devuelvan el car\'acter patog\'enico y de esa manera
podr\'ian producir enfermedad en el vacunado o sus contactos.
    
    Para lograr su objetivo, vac-o debe ser capaz de encontrar las variantes del
virus con menor probabilidad de revertir a la patogenia. Partiendo de la
secuencia gen\'omica que se encuentra en la cepa vacunal, se deben buscar las
modificaciones de esta secuencia que cumplan las siguientes propiedades:
    \begin{itemize}
      \item Que originen una respuesta inmune protectora contra el virus
salvaje.
      \item Que tengan muy baja probabilidad de mutar hasta convertirse en un
virus pat\'ogeno.
    \end{itemize}
    
    Entre todas las secuencias encontradas, vac-o debe construir una lista o
\textit{ranking} de secuencias con sus respectivos puntajes. Para lograr esto,
se deber\'a implementar una extensi\'on que brinde la informaci\'on y las
caracter\'isticas propias del virus. En particular, cada extensi\'on debe
especificar las regiones combinatorias de inter\'es y ser capaz de asignar un
puntaje a cada secuencia compar\'andola con la secuencia original. 

    En la Figura~\ref{fig:1} se presenta el flujo de trabajo y funcionamiento
general de vac-o. Notar que este esquema es puramente conceptual y no hace
ninguna referencia a los detalles de implementaci\'on. La intenci\'on es
clarificar al lector los diferentes ``actores'' y sus responsabilidades.
    \begin{figure}
	    \centering
	    \includegraphics[scale=0.65]{seq-diagram.png}
	    \caption{Flujo de trabajo de vac-o}
	    \label{fig:1}
    \end{figure}

    \subsubsection{Interfaces del Sistema}
    No se registran requerimientos.

    \subsubsection{Interfaces de Usuario}
    La interfaz con el usuario final consiste de dos formularios y un listado
con las secuencias resultantes. Inicialmente, el usuario de vac-o deber\'a
indicar la ubicaci\'on o \textit{path} de la extensi\'on que desea usar. El
siguiente paso, ser\'a completar un formulario con los datos requeridos por la
extensi\'on elegida. Finalmente, se da comienzo a la ejecuci\'on y se recibe
como resultado, la lista de secuencias con sus respectivos puntajes.

    \subsubsection{Interfaces de Hardware}
    No se registran requerimientos.

    \subsubsection{Interfaces de Software}
    \begin{itemize}
      \item BioPP: Se debe utilizar la librer\'ia BioPP para realizar el
\textit{folding} de secuencias gen\'omicas, predicci\'on de la estructura
secundaria y c\'alculo de distancias.
      \item Boost.Python: Se debe utilizar Boost.Python para brindar a los
desarrolladores de extensiones, la posibilidad de hacerlo utilizando el lenguaje
de programaci\'on Python.      
    \end{itemize}

    \subsubsection{Interfaces de Comunicaciones}
    No se registran requerimientos.

    \subsubsection{Restricciones de Memoria}
    No se registran restricciones de memoria para la ejecuci\'on del software.
No obstante, se debe tener en cuenta que dada la naturaleza y la complejidad del
problema, el tiempo de c\'alculo estar\'a directamente relacionado con la
memoria disponible.

    \subsubsection{Operaciones}
    No se registran requerimientos.

    \subsubsection{Requerimientos de Instalaci\'on}
    No se registran requerimientos.

  \subsection{Funciones del Producto}
  
  \subsubsection{Configuraci\'on inicial}
  Inicialmente, vac-o presenta al usuario final una pantalla con un formulario
donde el usuario debe indicar la ubicaci\'on de la extensi\'on que desea usar. A
continuaci\'on, vac-o carga la extensi\'on en memoria y presenta al usuario otro
formulario con los campos requeridos por la extensi\'on cargada.
  Una vez que los datos son validados, vac-o configura la extensi\'on con los
valores enviados por el usuario y queda listo para comenzar la ejecuci\'on
cuando el usuario lo indique.

  \subsubsection{Configuraci\'on de vac-o}
  Antes de dar comienzo a la generaci\'on de secuencias gen\'omicas, vac-o debe
solicitar a la extensi\'on cargada los siguientes valores:
  \begin{itemize}
    \item Secuencia inicial: La secuencia gen\'omica que se encuentra en la cepa
vacunal y se desea mejorar.
    \item Regiones combinatorias: Las regiones de la secuencia inicial que
resultan de inter\'es, indicando sus posiciones de inicio y fin, y qu\'e tipo de
regiones son cada una (restricciones en base a c\'odigo gen\'etico, o en base a
estructura secundaria).
    \item Estrategias de b\'usqueda: La estrategia que se debe utilizar para la
generaci\'on o b\'usqueda de nuevas secuencias.
  \end{itemize}
  
  \subsubsection{Generaci\'on de secuencias}
  Luego de las configuraciones b\'asicas, vac-o esta en condiciones de generar
nuevas secuencias gen\'omicas que cumplan las restricciones que hayan sido
impuestas. Para esto, se deben calcular las posibles ``variantes'' de cada
regi\'on combinatoria (teniendo en cuenta el tipo de regi\'on y sus posibles
intersecciones) y de acuerdo a la estrategia de b\'usqueda, construir nuevas
secuencias que ser\'an evaluadas posteriormente por la extensi\'on. Dado que el
espacio de b\'usqueda podr\'ia ser eventualmente muy grande, el criterio de
parada para la generaci\'on de secuencias tambi\'en debe ser provisto por la
extensi\'on.

  Notar que esta funcionalidad es el ``coraz\'on'' de vac-o y es donde se
encuentra la mayor complejidad del problema a resolver. 

  Supongamos $N \geqslant 1$ y $R_{1}, \cdots, R_{N}$ regiones combinatorias.
Luego, tendremos: \\
  \begin{center}    
    $M_{1\cdot1},..., M_{1\cdot j_{1}}$ mutaciones de $R_{1}$\\ 
    $M_{2\cdot1},..., M_{2\cdot j_{2}}$ mutaciones de $R_{2}$\\
    $\cdots$ \\
    $M_{N\cdot1},..., M_{N\cdot j_{N}}$ mutaciones de $R_{N}$   
  \end{center}
  con $j_{1}, \cdots, j_{N} \geqslant 1$. Es decir, que el n\'umero de posibles
secuencias que vac-o ``deber\'ia'' evaluar sera igual al producto $j_{1} \times
j_{2} \times \cdots \times j_{N}$. 

  Haciendo uso de diferentes estrategias de b\'usqueda, combinadas con las
funciones de evaluaci\'on provistas por la extensi\'on, vac-o intentara
optimizar esta generaci\'on de secuencias.

  \subsubsection{Evaluaci\'on de secuencias}
  De acuerdo a los puntajes asignados por la extensi\'on a cada una de las
secuencias generadas, vac-o debe construir un listado o \textit{ranking} de
secuencias que ser\'a dado al usuario como resultado final de la ejecuci\'on.

  \subsection{Caracter\'isticas de Usuarios}
  Se identifican 3 tipos de usuarios de vac-o:
  \begin{itemize}
    \item Final: que solo interact\'ua a trav\'es de la interfaz gr\'afica.
    \item Extensionista: que posee los conocimientos de programaci\'on
suficientes como para implementar extensiones de vac-o. A los fines de ampliar
los potenciales usuarios dentro de esta categor\'ia, es que se ofrece la
posibilidad de desarrollar extensiones usando el lenguaje Python.
    \item Contribuidor: que contribuye al c\'odigo fuente de vac-o realizando
mejoras o desarrollando nuevas funcionalidades.
  \end{itemize}

  \subsection{Restricciones}
  El producto debe ser desarrollado utilizando el lenguaje de programaci\'on C++
y bajo la licencia de software GPLv3.
  
  \subsection{Suposiciones y Dependencias}
    \begin{itemize}
      \item Sistema operativo: GNU/Linux
    \end{itemize}
  
  \subsection{Trabajo Futuro}
  Probablemente una de las principales mejoras a la primer versi\'on de vac-o,
ser\'a realizar las modificaciones necesarias para permitir la ejecuci\'on del
software en paralelo y de esta manera reducir los tiempos de c\'alculo. 
  Por otro lado, se deber\'a profundizar en el desarrollo de extensiones para
diferentes tipos de vacunas y virus.
  

\section{Requerimientos}
  \label{section-req}  
  \subsection{Funciones del Sistema}

  \subsubsection{Configuraci\'on inicial}
  El objetivo de esta funci\'on es la carga en memoria de la extensi\'on que se
desea utilizar para ejecutar vac-o.
  \begin{enumerate}
    \item \textbf{Carga de la extensi\'on:}
    El sistema recibe la ubicaci\'on del archivo \textit{.so} y lo carga en
memoria. Si la carga es exitosa, se devuelven los par\'ametros requeridos por la
extensi\'on (\textit{[(nombre, tipo, validaciones, defecto)]}) para su posterior
configuraci\'on. En caso de que la carga de la extensi\'on fracase, se devuelve
un mensaje de error.
    
    \item \textbf{Validaci\'on de valores:}
    La interfaz de usuario es responsable de la validaci\'on de los datos
ingresados por el usuario para la configuraci\'on de la extensi\'on. Para esto,
se deber\'an usar los diferentes \textit{criterios de validaci\'on} para cada
par\'ametro y solo cuando la validaci\'on sea exitosa, los valores son enviados
al sistema. Caso contrario, se debe presentar un mensaje de error al usuario.
    
    \item \textbf{Configuraci\'on de la extensi\'on:}
    El sistema recibe de la interfaz de usuario, la lista de par\'ametros
requeridos por la extensi\'on (\textit{[(nombre, valor)]}) y se asignan los
valores en la extensi\'on. Finalmente, se devuelve un mensaje de \'exito
indicando que vac-o est\'a listo para comenzar la ejecuci\'on.

    \item \textbf{Tama\~no del ranking:}
    El sistema recibe de la interfaz de usuario, el tama\~no del ranking de
secuencias que dar\'a como resultado de la ejecuci\'on.
  \end{enumerate}

  \subsubsection{Configuraci\'on de vac-o}
  El objetivo de esta funci\'on es solicitar a la extensi\'on cargada en
memoria, la informaci\'on necesaria para comenzar la ejecuci\'on del motor
combinatorio.
  \begin{enumerate}
    \item \textbf{Secuencia inicial:}
    El sistema solicita a la extensi\'on cargada en memoria, la secuencia de ARN
que se desea optimizar. La extensi\'on debe devolver la secuencia y el sistema
la almacena en memoria. 
    
    \item \textbf{Regiones combinatorias:}
    El sistema solicita a la extensi\'on cargada en memoria, las regiones
combinatorias de inter\'es para la optimizaci\'on. La extensi\'on debe devolver
las regiones (\textit{[(inicio, fin, tipo, evaluador)]}) y el sistema inicializa
cada regi\'on en memoria. Para cada regi\'on, el par\'ametro \textit{tipo} debe
ser uno de los siguientes:
    \begin{itemize}
      \item Estructura secundaria (SS): Las posibles variantes de la regi\'on
deben conservar la estructura secundaria.
      \item C\'odigo gen\'etico (GC): Las posibles variantes de la regi\'on
deben resultar silentes en t\'erminos de expresi\'on aminoc\'idica.
      \item Personalizada (CU): La extensi\'on debe proporcionar la
implementaci\'on de la regi\'on combinatoria.
    \end{itemize}

    Adem\'as, se debe considerar la posibilidad de que en el futuro se necesiten
nuevos tipos de regiones combinatorias.

    El par\'ametro \textit{evaluador} debe ser una funci\'on $f: Secuencia
\rightarrow (0,1)$ que ser\'a utilizada por el sistema para determinar
localmente la ``bondad'' de las diferentes variantes de cada regi\'on.

    \item \textbf{Estrategia de b\'usqueda:}
    El sistema solicita a la extensi\'on un \textit{umbral} de ``bondad'' y debe
implementar diferentes estrategias de optimizaci\'on, asegurando que se cumpla
lo siguiente:

    Sea $n$ el numero de regiones combinatorias, $s_{i}$ la secuencia
seleccionada de la regi\'on $i$ y $f_{i}: Secuencia \rightarrow (0,1)$ la
funci\'on de evaluaci\'on de la regi\'on $i$, con $i=1, ..., n$.
    \begin{center}
    $\prod_{i=1}^{n} f_{i}(s_{i}) \ge umbral$
    \end{center}

    \item \textbf{Regiones de validaci\'on:}
    El sistema solicita a la extensi\'on cargada en memoria, las regiones sobre
las que se debe hacer el ``control de calidad''. La extensi\'on debe devolver
las regiones (\textit{[(inicio, fin, prueba, criterio)]}) y el sistema
inicializa cada regi\'on en memoria. Para cada regi\'on, el par\'ametro
\textit{prueba}, debe ser uno de los siguientes:
      \begin{itemize}
        \item Mutaciones sistem\'aticas (ALL): Se calculan todas las posibles
mutaciones.
	\item Mutaciones al azar (RAND): Se calculan $N$ mutaciones al azar.
	\item Mutaciones personalizadas (CU): La extensi\'on debe proporcionar
la implementaci\'on.
      \end{itemize}    
    El par\'ametro \textit{criterio} debe ser uno de los siguientes:
      \begin{itemize}
        \item Similitud estructural (SS): La estructura secundaria de las
mutaciones debe tener un porcentaje de similitud mayor o igual que un valor
dado.
	\item Disimilitud estructural (DS): La estructura secundaria de las
mutaciones debe tener un porcentaje de similitud menor o igual que un valor
dado.
      \end{itemize}
    
    Adem\'as la extensi\'on debe especificar la profundidad $M$ con la que se
desea realizar el ``control de calidad''.    
  \end{enumerate}

  \subsubsection{Generaci\'on de secuencias}
  El objetivo de esta funci\'on es la construcci\'on de nuevas secuencias
gen\'omicas que cumplan con las restricciones impuestas.
  \begin{enumerate}
    \item \textbf{Generar variantes de regi\'on combinatoria:}
    Para cada regi\'on combinatoria, el sistema debe ser capaz de generar las
posibles variantes que cumplan con las restricciones impuestas por el tipo
de regi\'on y considerando las posibles intersecciones con las dem\'as regiones.
Adem\'as, debe ser posible el uso de diferentes librer\'ias externas a ser
determinadas por la extensi\'on.
    
    \item \textbf{Generar secuencias:}
    A partir de las variantes de cada regi\'on combinatoria, se deben construir
nuevas secuencia gen\'omicas, reemplazando en la secuencia inicial, cada
regi\'on por su variante.

    Esta funcionalidad representa el coraz\'on del sistema y consiste en la
construcci\'on de un \'arbol de (sub)secuencias derivadas de cada una de las
regiones combinatorias, donde cada hoja del \'arbol determinar\'a una nueva
secuencia gen\'omica completa. Luego, el sistema debe ser capaz de generar y
recorrer el \'arbol bas\'andose en la informaci\'on brindada por la extensi\'on.

    El criterio de terminaci\'on en la generaci\'on de secuencias (recorrido del
\'arbol) debe ser provisto por la extensi\'on.

    \item \textbf{Validar secuencias:}
    Cada secuencia generada por el sistema debe ser sometida al ``control de
calidad'' y solo aquellas que lo pasen exitosamente ser\'an consideradas para
una posterior evaluaci\'on.    

    El procedimiento consiste en generar y recorrer un \'arbol de mutaciones
acumuladas por cada ``regi\'on de validaci\'on'', que tendr\'a a lo sumo,
profundidad $M$ (dado por la extensi\'on). 

    Diremos que una regi\'on pasa el ``control de calidad'' cuando sea posible
generar el \'arbol completo. Es decir, cuando todos los nodos (mutantes) del
\'arbol, pasen el criterio de prueba especificado. Si alguna regi\'on no pasa el
``control de calidad'', entonces la secuencia se descarta.

    Tanto el procedimiento para generar las mutaciones en cada nivel del
\'arbol, como el criterio de prueba al que se somete cada mutaci\'on, est\'an
determinados por la ``regi\'on de validaci\'on''.
  \end{enumerate}

  \subsubsection{Evaluaci\'on de secuencias}
  El objetivo de esta funci\'on es la evaluaci\'on y posterior construcci\'on de
un \textit{ranking} de secuencias gen\'omicas.
  \begin{enumerate}    
    \item \textbf{Asignar puntaje a una secuencia:}
    El sistema solicita a la extensi\'on cargada en memoria, un puntaje para una
secuencia dada. La extensi\'on debe evaluar la secuencia recibida compar\'andola
con la secuencia pat\'ogena y devolver un puntaje.
    
    \item \textbf{Construir \textit{ranking} de secuencias:}
    El sistema debe construir un \textit{ranking} de secuencias bas\'andose en
los puntajes asignados por la extensi\'on. Este listado de secuencias es
adem\'as, la salida final de la ejecuci\'on.
  \end{enumerate}

  \subsection{Restricciones de Rendimiento}
  No se registran requerimientos.

  \subsection{Base de Datos}
  No se registran requerimientos.

  \subsection{Restricciones de Dise\~no}
    \subsubsection{Cumplimiento de Est\'andares}
    El producto debe cumplir con los siguientes principios de dise\~no de la
programaci\'on orientada a objetos. Los primeros 5, son tambi\'en conocidos por
el acr\'onimo ``\textbf{SOLID}''.
    \begin{itemize}
      \item \textbf{S}ingle responsibility principle (SRP)
      \item \textbf{O}pen/closed principle (OCP)
      \item \textbf{L}iskov substitution principle (LSP)
      \item \textbf{I}nterface segregation principle (ISP)
      \item \textbf{D}ependency inversion principle (DIP)   
      \item Law of Demeter (LoD)
    \end{itemize}

    Adem\'as el producto debe cumplir con el estandar ANSI C++ y el ``coding
style'' provisto por FuDePAN.

  \subsection{Atributos del Software}
    Se requiere que el c\'odigo del producto tenga un 80\% de cobertura con
pruebas automatizadas.

\pagebreak

\begin{appendices} 
  \section{Definiciones, Acr\'onimos y Abreviaturas}
  \label{appendix-def}
  \begin{itemize}
    \item \textbf{vac-o}: Combinatory Vaccine Optimizer.
    \item \textbf{FaMAF}: Facultad de Matem\'atica, Astronom\'ia y F\'isica.
    \item \textbf{UNC}: Universidad Nacional de C\'ordoba.
    \item \textbf{FuDePAN}: Fundaci\'on para el Desarrollo de la Programaci\'on en \'Acidos Nucleicos.
    \item \textbf{API}: Application Programming Interface.
    \item \textbf{GPL}: General Public License.
    \item \textbf{IEEE}: Institute of Electrical and Electronics Engineers
    \item \textbf{SOLID}: acr\'onimo nemot\'ecnico introducido por Robert C. Martin en la d\'ecada de 2000, que representa cinco principios b\'asicos de programaci\'on y dise\~no orientado a objetos.
    \item \textbf{LoD}: Law of Demeter, principio de dise\~no orientado a objetos para lograr ``bajo acoplamiento''.    
    \end{itemize}
  \section{Referencias}
  \label{appendix-ref}
  \begin{itemize}
    \item C++: Lenguaje de programaci\'on. \\
    \url{http://www.cplusplus.com}
    \item Python: Lenguaje de programaci\'on interpretado.\\ 
    \url{http://www.python.org}
    \item BioPP: Librer\'ia C++ para biolog\'ia molecular\\
    \url{http://code.google.com/p/biopp}
    \item Boost.Python: Librer\'ia C++ para la interacci\'on con Python.\\
    \url{http://www.boost.org/doc/libs/1_42_0/libs/python/doc/index.html}
    \item QT: Librer\'ia C++ para el desarrollo de interfaces gr\'aficas.\\
    \url{http://qt.nokia.com/products}
    \item GPL: General Public License. \\
    \url{http://www.gnu.org/licenses/gpl.html}
    \item IEEE STD 830-1998: Gu\'ia para la especifiaci\'on de requerimientos. \\
    \url{http://standards.ieee.org/reading/ieee/std_public/description/se/830-1998_desc.html}
    \item SOLID: ``Design Principles and Design Patterns'', Robert C. Martin. \\
    \url{http://www.objectmentor.com/resources/articles/Principles_and_Patterns.pdf}    
    \item FASTA: Formato basado en texto, utilizado para representar secuencias gen\'omicas.\\
    \url{http://es.wikipedia.org/wiki/Formato_FASTA}
  \end{itemize}
     
\end{appendices}

\end{document}
