\chapter{Todo concluye al fin}
\epigraph{Pase lo que pase, dirija quien dirija, todo el mundo sabe que la
camiseta 10 de la selecci\'on ser\'a m\'ia... Para siempre.}%
{Diego Armando Maradona}

\section{Aportes}

El principal aporte de este trabajo fue el an\'alisis y la formalizaci\'on del
problema biol\'ogico y la posterior propuesta de c\'omo solucionarlo
computacionalmente. En este sentido, se pudo implementar un software que sirve
como una prueba de concepto aunque todav\'ia queden funcionalidades y
consideraciones a tener cuenta antes de pensar en usarlo para el dise\~no de
vacunas atenuadas m\'as seguras.

Por otra parte, de alguna manera esto pretende ser un aporte (muy humilde por
cierto) a la ``interdisciplina'' como forma de aplicar los conocimientos
cient\'ificos en la resoluci\'on concreta de problemas que afectan la calidad de
vida de las personas.

Por \'ultimo, el trabajo fue presentado en la ``XXX Reuni\'on Cient\'ifica de la
\ac{SAV}'' donde recibimos muy buenas cr\'iticas y comentarios que
nos incentivan a seguir trabajando.

\section{Trabajo futuro}

Como dec\'iamos en la secci\'on anterior, queda pendiente continuar el
desarrollo del software para tener en cuenta:
\begin{itemize}
 \item Rango de temperaturas: esto afecta la predicci\'on de estructura
secundaria de \ac{RNA} y las probabilidades de mutaci\'on entre las bases de
nucle\'otidos, por lo que es muy relevante en la probabilidad de reversi\'on a
la virulencia.
 \item Soporte de recombinantes: se deber\'ia contemplar la probabilidad de que
el virus evolucione por combinaciones con virus hom\'ologos y evaluar las
repercusiones.
 \item Soporte para otros tipos de virus: tener en cuenta las particularidades
de otras clases de virus de la ``Clasificaci\'on de Baltimore''.
\end{itemize}

Desde un punto de vista m\'as computacional, se deber\'ia trabajar en:
\begin{itemize}
 \item Paralelizaci\'on: esto es fundamental para poder ejecutar el
software con datos reales, fundamentalmente debido al costo de realizar la
predicci\'on directa e inversa de la estructura secundaria de \ac{RNA}.
 \item Estrategias de b\'usqueda: se deber\'ia analizar la posibilidad de
implementar otros algoritmos de b\'usqueda local m\'as complejos y que
probablemente conduzcan a mejores resultados.
\end{itemize}

Tambi\'en queda pendiente la posibilidad de desarrollar extensiones usando
lenguajes de programaci\'on m\'as ``\'agiles'' que C++, como podr\'ia ser el
caso de Python y el desarrollo de una interfaz m\'as ``amigable'' para el
usuario final. 

Por \'ultimo, se deben realizar ensayos con datos reales que permitan
caracterizar de mejor manera el espacio de b\'usqueda y en funci\'on de eso,
determinar los par\'ametros que mejor se ajusten al mismo.
