\begin{abstract}

Las denominadas vacunas vivas o atenuadas han sido ampliamente utilizadas para 
prevenir enfermedades como la rubeola, la poliomielitis, el sarampi\'on o la
fiebre amarilla. Sin embargo, uno de los peligros potenciales del uso de este
tipo de vacunas es la probabilidad de reversi\'on a la virulencia, produciendo
la enfermedad que intentan prevenir.

En trabajos previos se han desarrollado varios enfoques para reducir la
probabilidad de reversi\'on, tales como priorizar el uso de determinados codones
o pares de codones, sin modificar la secuencia aminoac\'idica\cite{Coleman08} o
seleccionar polimerasas m\'as fidedignas\cite{Vignuzzi08}.

En este trabajo se propone un enfoque complementario, que consiste en maximizar
el n\'umero de mutaciones necesario para que la secuencia de \ac{RNA} del virus
vacunal llegue a convertirse en una secuencia semejante a las pat\'ogenas o
revertantes, siempre manteniendo las propiedades que le otorgan la atenuaci\'on.

Hemos analizado y formalizado una soluci\'on para este problema, y la hemos
implementado en una herramienta de software. Esta herramienta es altamente
configurable, con lo cual puede tratar diferentes virus. Como prueba de
concepto del software, se ha aplicado a la vacuna oral contra la poliomielitis.
En este caso, la atenuaci\'on viene dada por la estructura secundaria de su
\ac{RNA}, con lo cual se ha configurado el software espec\'ificamente para
conservar este aspecto.

\end{abstract}
