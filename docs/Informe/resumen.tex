\begin{abstract}

Las denominadas vacunas vivas o atenuadas, han sido ampliamente utilizadas para 
prevenir enfermedades como la viruela, la poliomielitis, el sarampi\'on o la
fiebre amarilla. Sin embargo, uno de los peligros potenciales del uso de este
tipo de vacunas es la probabilidad de reversi\'on a la virulencia, produciendo
la enfermedad que intentan prevenir.

Se han desarrollado varios enfoques para reducir la probabilidad de reversi\'on,
tales como priorizar el uso de determinados codones o pares de codones, sin
modificar la secuencia aminoac\'idica\cite{Coleman08} o seleccionar polimerasas
m\'as fidedignas\cite{Vignuzzi08}.

En este trabajo se propone un desarrollo complementario, que consiste en la
implementaci\'on de un software para maximizar el n\'umero de mutaciones
necesario para que la secuencia de \ac{RNA} de la vacuna, alcance secuencias
semejantes a las pat\'ogenas o revertantes, manteniendo las propiedades
que le otorgan la atenuaci\'on, poniendo especial hincapi\'e en la
conservaci\'on de la estructura secundaria.

\end{abstract}
