\chapter{Integraci\'on de librer\'ias externas}
\label{librerias}
\epigraph{In theory, there is no difference between theory and practice. But, in
practice, there is.}%
{Jan L. A. van de Snepscheut}

A continuaci\'on veremos los detalles de implementaci\'on m\'as relevantes de
las interfaces propuestas en la secci\'on~\ref{diseno-libRNA} y en particular
para los algoritmos de predicci\'on inversa de la estructura secundaria de
\ac{RNA}.

\section{Implementar vs. Integrar}

Ante el requerimiento de contar con algoritmos para realizar la comparaci\'on y
predicci\'on (directa e inversa) de la estructura secundaria de \ac{RNA},
debimos evaluar y decidir, entre implementar completamente los algoritmos o
integrar/adaptar los algoritmos existentes.

En este sentido, las interfaces propuestas en la secci\'on~\ref{diseno-libRNA}
no descartan ninguna de las dos posibilidades, sino simplemente especifican los
``servicios'' que deben ofrecer al sistema las eventuales implementaciones
(propias o ajenas).

Por otro lado, implementar este tipo de algoritmos requiere de un profundo
conocimiento del dominio, en este caso de qu\'imica molecular, que se desviaba
de los objetivos del software y del trabajo en general. Adem\'as, debido a la
complejidad de los mismos, se requieren validaciones emp\'iricas para demostrar
la fidelidad de los resultados lo cual exced\'ia el alcance de este trabajo.

Finalmente se opto por integrar al sistema los siguientes algoritmos para
implementar las respectivas interfaces:
\begin{itemize}
 \item \textbf{IFold:} RNAfold\cite{Hofacker94}.
 \item \textbf{IFoldInverse:} RNAinverse\cite{Hofacker94},
INFO-RNA\cite{Busch07}.
 \item \textbf{IStructureCmp:} RNAforester\cite{Hochsmann05}
\end{itemize}

N\'otese que no incluimos la interfaz \textbf{ISequenceCmp} debido a que la
comparaci\'on entre secuencias de \ac{RNA} es significativamente mas simple y
para este caso, se opto por realizar una implementaci\'on propia utilizando una
matriz de costos como se propuso en la secci\'on~\ref{formalizacion-prop}.

\section{Integraci\'on}

La integraci\'on de los programas RNAfold y RNAforester, no presentaron mayores
dificultades ya que en ambos casos, la implementaci\'on consiste simplemente 
en ejecutar el programa y luego leer el resultado. Los dos programas
realizan c\'alculos complejos y costosos computacionalmente pero
deterministicos. Por otro lado, los programas que realizan la predicci\'on
inversa de la estructura secundaria de \ac{RNA} o \textit{inverse folding}
(INFO-RNA y RNAinverse) presentan algunas complicaciones principalmente debido a
su no determinismo y que veremos a continuaci\'on.

\subsection{Predicci\'on inversa (\textit{inverse folding})}

Recordando lo dicho en la secci\'on~\ref{inverse}, el objetivo de los programas
(INFO-RNA y RNAinverse) que realizan el \textit{inverse folding} es encontrar
secuencias de \ac{RNA} que tengan la estructura secundaria de \ac{RNA} dada por
el usuario. El problema surge cuando se los quiere usar de manera sistem\'atica,
como es nuestro caso. Es decir, lo que pretendemos al usar estos programas es
tener la capacidad de analizar sistem\'aticamente, la mayor cantidad posible de
secuencias de \ac{RNA} que tengan una estructura secundaria de \ac{RNA}
determinada para luego evaluar cuales de esas secuencias contribuyen de mejor
manera a la atenuaci\'on del virus.

\subsubsection{Caracterizaci\'on de los algoritmos para la predicci\'on inversa}

M\'as all\'a de los diferentes par\'ametros que aceptan cada uno de estos
programas, comparten dos par\'ametros fundamentales. Por un lado, la estructura
secundaria de \ac{RNA} sobre la que se quiere hacer la predicci\'on inversa y
por otro, una secuencia de \ac{RNA} ``incompleta'' (algunas bases indefinidas)
que se usa como secuencia de inicio en el algoritmo de b\'usqueda. 

La secuencia ``incompleta'' que se da como par\'ametro, cumple adem\'as el rol
de especificar las restricciones que se deben satisfacer durante la b\'usqueda
(qu\'e bases de nucle\'otidos \textit{deben} estar presentes en determinadas
posiciones). Esto por un lado restringe el espacio de b\'usqueda, pero por otro
lado se podr\'ian estar especificando restricciones que no se pueden satisfacer
para la estructura secundaria de \ac{RNA} dada. En nuestro caso de uso,
\ac{vac-o} comienza su ejecuci\'on con una secuencia de \ac{RNA} (el genoma del
virus atenuado) que efectivamente tiene la estructura secundaria de \ac{RNA} que
se quiere conservar. Por lo tanto, esta misma secuencia se puede usar para
generar distintas secuencias ``incompletas'' con la garant\'ia de que todas
especifiquen restricciones satisfacibles.

El no determinismo est\'a dado por el hecho de que para una misma estructura
secundaria de \ac{RNA} y una misma secuencia (incompleta) de inicio, en
sucesivas ejecuciones de los programas, se pueden encontrar distintas secuencias
de \ac{RNA} aunque eventualmente tambi\'en repetidas. Por otro lado, modificar
la secuencia (incompleta) de inicio, podr\'ia aunque no necesariamente, implicar
que se encuentren nuevas secuencias.

Con todo lo dicho, deber\'ian quedar claras las complicaciones que presentan
estos programas para ser ejecutados de manera sistem\'atica. Pero para dejarlo
del todo claro, podemos formalizarlo de la siguiente manera.

Recordemos que una secuencia de \ac{RNA} se representa como una palabra sobre el
alfabeto $\Sigma = \{A,U,G,C\}$ y una secuencia de \ac{RNA} ``incompleta'' como
una palabra sobre el alfabeto $\Sigma \cup \{N\}$ donde las $N$ representan las
bases indefinidas. Adem\'as, si $s$ es una secuencia de \ac{RNA} (completa o
incompleta) de longitud $n$, $s_{i}$ sera la posici\'on o base $i$-esima de la
secuencia $s$ con $0 \leq i < n$.

Luego, si definimos:
\begin{itemize}
 \item $\mathcal{S}$ el conjunto de secuencias de \ac{RNA} de longitud $n$.
 \item $\mathcal{E}$ el conjunto de estructuras secundarias de \ac{RNA} sobre
secuencias de longitud $n$. 
 \item $\mathcal{I}(e) \subseteq \mathcal{S}$ el conjunto de secuencias de
\ac{RNA} que tienen como estructura secundaria a $e \in \mathcal{E}$.
 \item $\mathcal{R}$ el conjunto de secuencias de \ac{RNA} incompletas de
longitud $n$.
\end{itemize}

podemos caracterizar a los algoritmos para la predicci\'on inversa de la
estructura secundaria de \ac{RNA} como $inverse: \mathcal{E} \times \mathcal{R}
\rightarrow \mathcal{S}$ tal que para $e \in \mathcal{E}$ y $r \in
\mathcal{R}$, si $inverse(e,r) = s$ entonces se satisfacen:
\begin{itemize}
 \item $s \in \mathcal{I}(e)$
 \item $\{\forall i : 0 \leq i < n \land r_{i} \neq N : s_{i} = r_{i}\}$ 
\end{itemize}

El problema es b\'asicamente que no conocemos la cardinalidad de
$\mathcal{I}(e)$ y que adem\'as, si pensamos a las sucesivas ejecuciones de
$inverse(e,r)$ como una forma de enumerar el conjunto $\mathcal{I}(e)$, el orden
en que se enumeran las secuencias ser\'ia aleatorio y permitir\'ia repeticiones
(no deterministico).

\subsubsection{Propuesta para sistematizar la predicci\'on inversa}

Vistas las complicaciones para lograr analizar sistem\'aticamente todas, o al
menos un gran numero de secuencias que tengan una estructura secundaria de
\ac{RNA} determinada, veremos de que manera integramos al sistema los algoritmos
mencionados.

La idea consiste b\'asicamente en ``recordar'' las secuencias encontradas en
ejecuciones anteriores y en sistematizar la forma de generar secuencias
incompletas. Tener ``memoria'' sobre las secuencias encontradas es fundamental
para no volver a evaluar la misma secuencia. Por otro lado, sistematizar la
generaci\'on de secuencias incompletas tiende a restringir el espacio de
b\'usqueda desde diferentes ``frentes'' con el objetivo de alcanzar a cubrir la
mayor cantidad del espacio de b\'usqueda posible.

Para generar secuencias incompletas se propuso recorrer iterativamente todas
las formas de seleccionar $k$ posiciones de $n$, donde $n$ es la longitud de la
secuencia completa y $k$ es un par\'ametro determinado por la extensi\'on. Es
decir que tendremos $n \choose k$ secuencias incompletas distintas. Sin
embargo, aun no tenemos forma de determinar cuando se deber\'ia pasar a la
siguiente secuencia de inicio ya que el hecho de que el algoritmo de
predicci\'on inversa, devuelva una secuencia que ya hab\'ia sido encontrada
previamente, no implica que en sucesivas ejecuciones y sin cambiar la secuencia
de inicio, se encuentren nuevas secuencias. Luego, se propuso delegar esta
responsabilidad a la extensi\'on y que sea esta quien determine el n\'umero $m$
de intentos que se realizan sobre cada secuencia de inicio hasta encontrar una
secuencia nueva.

A continuaci\'on presentamos el pseudo c\'odigo de la propuesta asumiendo las
siguientes variables:
\begin{itemize}
 \item $e \in \mathcal{E}$: la estructura secundaria de \ac{RNA} sobre la que se
realiza la predicci\'on inversa.
 \item $i \in \mathcal{I}(e)$: la secuencia de \ac{RNA} a la que se le desean
buscar variantes que conserven la estructura secundaria $e$.
 \item $n \in \mathbb{N}$: la longitud de la secuencia $i$.
 \item $r \in \mathcal{R}$: la secuencia incompleta de inicio.
 \item $k \in \mathbb{N}$: el n\'umero de posiciones indefinidas para generar
secuencias incompletas.
 \item $m \in \mathbb{N}$: el n\'umero de intentos por cada secuencia incompleta
de inicio.
 \item $f \subseteq \mathcal{I}(e)$: el conjunto de secuencias encontradas.
\end{itemize}

y los siguientes procedimientos auxiliares:
\begin{itemize}
 \item $inverse(e,r)$: algoritmo externo como se caracterizo anteriormente.
 \item $begin(i,n,k)$: devuelve la primera de las $n \choose k$ secuencias
incompletas a partir de la secuencia $i$.
 \item $next(r)$: devuelve la siguiente secuencia incompleta, para alg\'un orden
de las $n \choose k$ secuencias incompletas a partir de la secuencia $i$.
\end{itemize}


\begin{algorithm}
 \caption{Inicializaci\'on}
\begin{algorithmic}[1]
\STATE $f \leftarrow \emptyset$ 
\STATE $r \leftarrow begin(i,n,k)$
\end{algorithmic}
\end{algorithm}

\begin{algorithm}
 \caption{Predicci\'on inversa integrada}
 \begin{algorithmic}[1]
 \STATE $j \leftarrow m$
 \REPEAT
 \STATE $j \leftarrow j-1$
 \STATE $s \leftarrow inverse(e,r)$
 \IF{$s \in f$ \textbf{and} $j = 0$}
 \STATE $r \leftarrow next(r)$
 \STATE $j \leftarrow m$
 \ENDIF
 \UNTIL{$s \notin f$}
 \STATE $f \leftarrow f \cup \{s\}$
 \RETURN{s}
 \end{algorithmic}
\end{algorithm}
