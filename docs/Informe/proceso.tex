\chapter{Proceso de desarrollo}
\epigraph{How does a project get to be a year late?... One day at a time.}%
{Fred Brooks}

A esta altura ya hemos presentado los principales conceptos biol\'ogicos
involucrados en este trabajo, el dise\~no de vacunas atenuadas y la
formalizaci\'on del problema que nos propusimos resolver. En los cap\'itulos
restantes veremos los aspectos t\'ecnicos mas relevantes sobre el desarrollo del
software.

\section{Modelo de desarrollo}

Para llevar adelante el desarrollo del software, se opto por el cl\'asico
modelo de cascada. Fundamentalmente debido a su simplicidad y a que los
requerimientos con que deb\'ia cumplir el software estaban bien definidos desde
el inicio. No solo a nivel funcional, sino tambi\'en en cuanto a principios del
dise\~no orientado a objetos.

\subsection{Etapas de la cascada}

Las etapas que se llevaron a cabo durante el desarrollo de este trabajo y que
veremos con mayor detenimiento en los siguientes cap\'itulos, fueron las
siguientes:
\begin{enumerate}
 \item Especificaci\'on de requerimientos.
 \item Dise\~no.
 \item Implementaci\'on.
 \item Verificaci\'on.
\end{enumerate}

Quedaron fuera del alcance, las etapas ``Instalaci\'on'' y por supuesto,
``Mantenimiento''.

\section{Ecosistema}

El ``ecosistema'' de herramientas que se utilizaron para llevar adelante el
proceso de desarrollo son las siguientes:

\begin{itemize}
 \item \textbf{Lenguaje de programaci\'on:} El software se implemento en
\textbf{C++}\footnote{\url{http://cplusplus.com}}.
 \item \textbf{Lenguaje de dise\~no:} El dise\~no del software se hizo
utilizando \textbf{UML}\footnote{\url{http://www.uml.org}}.
 \item \textbf{Control de versiones:} Se utilizo el sistema de control de
versiones \ac{SVN}\footnote{\url{http://subversion.apache.org}} y puede ser
consultado en \url{http://vac-o.googlecode.com}.
 \item \textbf{Sistema de ``construcci\'on'':} Para automatizar el
proceso de compilaci\'on del c\'odigo fuente se
utilizo \textbf{CMake}\footnote{\url{http://www.cmake.org}}.
 \item \textbf{Automatizaci\'on de pruebas:} Para realizar la verificaci\'on del
software, se opto por implementar pruebas
unitarias utilizando
\textbf{google-test}\footnote{\url{http://googletest.googlecode.com}} y
\textbf{google-mock}\footnote{\url{http://googlemock.googlecode.com}}. 
 \item \textbf{An\'alisis est\'atico:} Se utilizaron las herramientas
\textbf{astyle}\footnote{\url{http://astyle.sourceforge.net}} y
\textbf{cppcheck}\footnote{
\url{http://sourceforge.net/apps/mediawiki/cppcheck}}.
 \item \textbf{An\'alisis din\'amico:} Se utilizaron las herramientas
\textbf{valgrind}\footnote{\url{http://valgrind.org}} y
\textbf{gcov}\footnote{\url{http://gcc.gnu.org/onlinedocs/gcc/Gcov.html}} para
verificar la ausencia de ``memory leaks'' y la cobertura de las pruebas.
\end{itemize}
