\chapter{Dise\~no de vacunas atenuadas}
\label{diseno}
\epigraph{Science is always wrong. It never solves a problem without creating
ten more.}%
{George Bernard Shaw}

Una vacuna atenuada es aquella que es creada reduciendo la virulencia de un
pat\'ogeno pero aun as\'i, manteni\'endolo viable (``vivo''). Antes de
adentrarnos en los detalles de lo que plantea este trabajo como metodolog\'ia
para racionalizar el dise\~no de vacunas atenuadas, veremos brevemente algunos
antecedentes y cual fue, y sigue siendo, la metodolog\'ia cl\'asica para la
producci\'on de este tipo de vacunas.

\section{Dise\~no cl\'asico}

La metodolog\'ia para la producci\'on de vacunas atenuadas ha sido
hist\'oricamente, el pasaje del virus a trav\'es de cultivos de c\'elulas
distintas a las c\'elulas hu\'esped. De esta manera, el virus tiende a
``evolucionar'' para adaptarse al nuevo hu\'esped y ser capaz de reproducirse.

Este concepto de ``evoluci\'on'', se traduce en alguna cantidad de mutaciones
sobre la secuencia de nucle\'otidos del virus y que se espera, reduzcan su
capacidad de reproducirse en el hu\'esped original. Luego, son precisamente
estas mutaciones las que le confieren la atenuaci\'on y dan lugar a la vacuna
atenuada.

La principal desventaja en este proceso, es que las mutaciones que se producen
son totalmente impredecibles y aun cuando estas mutaciones derivan en un virus
atenuado, este podr\'ia revertir muy f\'acilmente a la virulencia dependiendo de
la naturaleza de las mutaciones que generan la atenuaci\'on\cite{Badgett02}.
Adem\'as, como vimos en la secci\'on~\ref{virus}, la alta frecuencia de
mutaciones que poseen los virus \ac{RNA} aumenta la probabilidad de reversi\'on
a la virulencia. De hecho, esto es lo que ocurre con muchas vacunas atenuadas y
en particular, con la \ac{OPV}.

A pesar de este peligro de reversi\'on a la virulencia, esta sigue siendo la
principal metodolog\'ia para la producci\'on de vacunas atenuadas. Esto se debe,
fundamentalmente, a la falta de conocimiento sobre el significado o incidencia
de las mutaciones en la atenuaci\'on de un determinado virus. Sin embargo, los
recientes avances en la virolog\'ia molecular, han permitido explorar nuevas
t\'ecnicas que permitan controlar la replicaci\'on de un virus o su virulencia y
esto abri\'o la puerta a lo que se denomina ``dise\~no racional de vacunas
atenuadas''\cite{Lauring10}.

\section{Dise\~no racional}

La idea central en esta nueva metodolog\'ia, en la que se enmarca este trabajo,
es explotar el conocimiento que se ha producido en los \'ultimos a\~nos acerca
de la biolog\'ia molecular de determinados virus. Pudiendo controlar la
replicaci\'on de un virus o su virulencia, ser\'ia posible dise\~nar vacunas
atenuadas ``seguras'' evitando la impredecibilidad de las atenuaciones
emp\'iricas obtenidas mediante el dise\~no cl\'asico.

\subsection{Antecedentes}

Existen diferentes aproximaciones al dise\~no racional de vacunas
atenuadas\cite{Lauring10}. En general, todas estas aproximaciones se encuentran
en fase experimental y todav\'ia no se han aplicado en producci\'on. A
continuaci\'on hacemos menci\'on a tan solo dos de ellas, fundamentalmente para
marcar la diferencia con el dise\~no cl\'asico que presentamos anteriormente.

\subsubsection{Fidelidad en la replicaci\'on\cite{Vignuzzi08}} 

Como ya vimos en la secci\'on~\ref{virus}, la alta frecuencia de mutaciones en
los virus \ac{RNA} se debe a la alta tasa de error en su \ac{RNA} polimerasa.
Luego, modificando la \ac{RNA} polimerasa de tal manera que se reduzca su tasa
de error, se obtendr\'ia un virus atenuado mas estable y con menor probabilidad
de revertir a la virulencia en las sucesivas replicaciones. 

La principal desventaja de esta aproximaci\'on radica en que las
posibles variantes sobre la \ac{RNA} polimerasa deben ser determinadas y
evaluadas experimentalmente para cada virus en particular.

\subsubsection{(De-)Optimizaci\'on de codones\cite{Mueller10, Coleman08}} 

Si recordamos lo dicho en la secci\'on~\ref{bio-esencial} sobre el c\'odigo
gen\'etico, vimos que cada amino\'acido puede ser codificado hasta por 6
codones distintos. Es decir, distintas secuencias de nucle\'otidos resultan
equivalentes en t\'erminos de los amino\'acidos que codifican. Concretamente,
una prote\'ina de 300 amino\'acidos puede ser codificada por aproximadamente, 
$10^{151}$ secuencias de nucle\'otidos. 

Sin embargo, experimentalmente se pudo comprobar que algunos codones son m\'as
frecuentes que otros (\textit{codon bias}). Similarmente, pero de manera
independiente, se comprob\'o que determinados pares de codones son m\'as
frecuentes que otros (\textit{codon pair bias}). Aunque todav\'ia no est\'a
claro a que se debe esta ``parcialidad'' en el uso de codones o pares de
codones, se supone que afectar\'ia el proceso de s\'intesis de prote\'inas
(traducci\'on). 

Lo que se propone con esta aproximaci\'on, es determinar las secuencias de
nucle\'otidos que conserven la secuencia aminoac\'idica del virus pero que al
mismo tiempo, tiendan a usar codones y pares de codones menos frecuentes. De
esta manera, la atenuaci\'on del virus se obtendr\'ia debilitando su capacidad
de traducci\'on y replicaci\'on. 

Entre las ventajas que presenta esta metodolog\'ia, se destaca por un lado que
la atenuaci\'on es el resultado de un an\'alisis sistem\'atico y por lo tanto,
aplicable a diferentes virus de manera autom\'atica. Por otro lado, la alta
cantidad de cambios que se realizan sobre el virus original sugieren una menor
probabilidad de revertir a la virulencia.

\section{Propuesta de soluci\'on}
\label{vacunas-propuesta}

En pocas palabras, la propuesta consiste en encontrar un conjunto de secuencias
de \ac{RNA} que conserven las propiedades que le otorgan la atenuaci\'on al
virus y que al mismo tiempo, tiendan a maximizar el numero de mutaciones
necesarias para alcanzar secuencias semejantes a las pat\'ogenas o revertantes.

Esto tiene algunos puntos en com\'un con la ``(de-)optimizaci\'on de codones''
que presentamos anteriormente. En particular, ambas aproximaciones comparten la
idea de sistematizar el dise\~no de forma tal que pueda ser usado para
diferentes virus. Esto implica fundamentalmente, plasmar la metodolog\'ia en la
implementaci\'on de un software que a partir de una serie de datos provistos por
el usuario, devuelva como resultado una o varias secuencias de nucle\'otidos que
representen posibles atenuaciones del virus.

De una forma mas abstracta, podemos pensar en un software para el dise\~no de
secuencias de nucle\'otidos basado en restricciones. Fundamentalmente, lo que se
busca es ``generar'' secuencias que satisfagan determinadas propiedades o
restricciones. En particular, aquellas que tiendan a reducir la virulencia del
virus.

En este sentido, se podr\'ia trazar una analog\'ia con los algoritmos para la
predicci\'on inversa de estructura secundaria (\textit{inverse folding}). En
esencia, estos algoritmos ``dise\~nan'' secuencias de \ac{RNA} que tengan como
estructura secundaria \ac{mfe}, la estructura dada por el usuario. De hecho,
como mencionamos en la secci\'on~\ref{inverse}, el problema se plantea
computacionalmente como un \ac{CSP}.

La principal innovaci\'on de esta propuesta es que las restricciones se enfocan
esencialmente en el \ac{IRES} y su estructura secundaria. Por lo visto en la
secci\'on~\ref{virus}, parece clara la importancia del \ac{IRES} en la
traducci\'on y posterior replicaci\'on de diferentes virus \ac{RNA} y en
particular del poliovirus.

Lo que nos proponemos en este trabajo, es realizar un an\'alisis sistem\'atico
de las posibles variantes al \ac{IRES} de los virus atenuados  Sabin (y
eventualmente cualquier otro) que conserven la estructura secundaria y
en consecuencia, la atenuaci\'on del virus. Luego, maximizando la cantidad de
mutaciones necesarias para revertir a secuencias semejantes a las pat\'ogenas o
revertantes, estar\'iamos reduciendo la probabilidad de que el virus atenuado
sufra reversi\'on a la virulencia.

Esquem\'aticamente, podr\'iamos plantear el problema de la siguiente manera:
\begin{itemize}
 \item \textbf{Entrada:} Genoma del virus atenuado, genomas de los pat\'ogenos
o revertantes y un conjunto de restricciones. Fundamentalmente, la
conservaci\'on de la estructura secundaria del \ac{IRES} del virus atenuado.
 \item \textbf{Objetivo:} Satisfaciendo las restricciones impuestas, maximizar
la distancia entre el genoma del virus atenuado y los genomas pat\'ogenos o
revertantes.
 \item \textbf{Salida:} Una o varias secuencias, candidatas a ``mejorar'' el
virus atenuado.
\end{itemize}

\subsection{Formalizaci\'on}

Como mencionamos en la secci\'on~\ref{propuesta} esto puede ser visto como un
problema de \textbf{``optimizaci\'on combinatoria basado en restricciones''}.
Pero para hacerlo, primero se deben identificar algunos elementos fundamentales
que permitan definir el problema.

Repasando lo dicho en la secci\'on~\ref{propuesta}, este tipo de problemas
consiste en asignar valores a un conjunto finito de variables (componentes de
una soluci\'on) que satisfagan determinadas restricciones. En nuestro caso,
estas restricciones ser\'an propiedades biol\'ogicas sobre partes de una
secuencia de \ac{RNA} y los posibles valores a asignar ser\'an (sub)secuencias
de \ac{RNA} que satisfagan las propiedades requeridas.

Sobre diferentes partes de la secuencia de \ac{RNA} se podr\'ian requerir
diferentes propiedades biol\'ogicas (restricciones). Luego, ser\'an estas
propiedades las que determinen los posibles valores sobre cada parte
``variable'' de la secuencia. Finalmente, las combinaciones de los posibles
valores para cada parte de la secuencia, formaran las potenciales soluciones del
problema.

\subsubsection{Definici\'on del problema}

Sea $N$ la longitud de la secuencia de \ac{RNA} del virus atenuado, y para $k
\in \mathbb{N}$ sea $\mathcal{S}_{k}$ el conjunto de secuencias de \ac{RNA} de
longitud $k$. Entonces definimos:

\begin{itemize} 
 \item \textbf{Espacio de soluciones:} $\mathcal{S}_{N}$

 \item \textbf{Componentes variables de una soluci\'on:} $s_{1},s_{2}, \dots,
s_{n}$ tal que $s_{i} \in \mathcal{S}_{N_{i}}$ con $1 \le i \le n$ y $0 < N_{i}
\le N$.

 \item \textbf{Restricciones sobre las componentes:} Conservaci\'on de la
estructura secundaria o de la secuencia aminoac\'idica con respecto al virus
atenuado. Eventualmente, se podr\'ian contemplar otras restricciones que
impliquen propiedades biol\'ogicas que resulten de inter\'es para la
atenuaci\'on del virus.

 \item \textbf{Funci\'on ``objetivo'' o de evaluaci\'on:} $f: \mathcal{S}_{N}
\rightarrow \mathbb{R}$ tal que $f(s)$ calcula la bondad de cada soluci\'on, en
nuestro caso, como la distancia en n\'umero de mutaciones necesarias para
llegar de $s$ a alguna secuencia pat\'ogena o revertante.
\end{itemize}

Lo primero que podemos mencionar es que para cualquier virus \ac{RNA}, recorrer
el espacio de soluciones de manera exhaustiva es inviable ya que por ejemplo,
para el caso del poliovirus, $N \simeq 7,500$. Es decir que existen
aproximadamente $4^{7,500}$ posibles secuencias de \ac{RNA} (7,500
posiciones y 4 bases de nucle\'otidos posibles para cada posici\'on).

Por otro lado, la posibilidad de evaluar los posibles valores para cada
componente $s_{i}$ de manera exhaustiva, depender\'a fundamentalmente del tipo
de restricci\'on impuesta sobre esa componente y de la longitud $N_{i}$. En
particular para los virus atenuados Sabin y la conservaci\'on de la estructura
secundaria del \ac{IRES} ($N_{i} \simeq 400$), la evaluaci\'on exhaustiva
ser\'ia inviable debido al alto costo de predecir la estructura secundaria,
como vimos en las secciones~\ref{folding} y \ref{inverse}.

Con el problema planteado de esta manera y debido a la inviabilidad de realizar
una b\'usqueda exhaustiva, se pueden utilizar diferentes algoritmos de
b\'usqueda local para que partiendo de una secuencia de \ac{RNA} inicial, en
este caso el virus atenuado, recorrer el espacio de b\'usqueda $\mathcal{S}_{N}$
teniendo como objetivo, maximizar la funci\'on de evaluaci\'on $f$.

En este contexto, la definici\'on de la funci\'on de evaluaci\'on $f:
\mathcal{S}_{N} \rightarrow \mathbb{R}$ sera crucial para encontrar buenas
soluciones. En principio se podr\'ia pensar que la imagen de la funci\'on, sea
$\mathbb{N}$ en lugar de $\mathbb{R}$. De hecho, este es el caso si la funci\'on
calcula la distancia de \textit{Hamming} est\'andar. Esto es, sumar 1 por cada
una de las bases en las que difiere la secuencia soluci\'on de la secuencia
pat\'ogena. Pero de esta manera se estar\'ia suponiendo que la probabilidad de
mutaci\'on entre las bases es uniforme y esto no es as\'i necesariamente. 

Luego, determinando emp\'iricamente la probabilidad de mutaci\'on de cada base
hacia cualquiera de las otras tres, se podr\'ia incluir esta informaci\'on en la
funci\'on de evaluaci\'on usando una matriz de dimensi\'on $4\times4$ indicando
en cada posici\'on $(i,j)$, el costo de realizar la mutaci\'on de la base $i$ a
la base $j$.

En particular, para la distancia de \textit{Hamming} podemos definir la
siguiente matriz de costos:
\[
M = 
\begin{Bmatrix}
 0 & 1 & 1 & 1\\
 1 & 0 & 1 & 1\\
 1 & 1 & 0 & 1\\
 1 & 1 & 1 & 0
\end{Bmatrix}
\]

