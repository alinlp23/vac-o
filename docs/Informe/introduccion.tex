\chapter{Introducci\'on}
\epigraph{A foolish faith in authority is the worst enemy of truth.}%
         {Albert Einstein}

\section{Para los ansiosos}
El objetivo de este trabajo es el dise\~no y desarrollo de un software que
sirva como soporte para el dise\~no de vacunas atenuadas. En este sentido, la
propuesta es encontrar un conjunto de secuencias de \ac{RNA} que conserven las
propiedades que le otorgan la atenuaci\'on a la vacuna y que al mismo tiempo,
tiendan a maximizar el numero de mutaciones necesarias para alcanzar secuencias
semejantes a las pat\'ogenas o revertantes\footnote{Para el lector ajeno a la
biolog\'ia, qu\'e producen la enfermedad que la vacuna debiera prevenir.}. 

En los cap\'itulos~\ref{biologia} y \ref{bioinformatica} se introducen los
conceptos biol\'ogicos b\'asicos y algunos de los problemas caracter\'isticos
de la bioinform\'atica, profundizando en aquellos que est\'an mas relacionados
con este trabajo. Luego, en el cap\'itulo~\ref{diseno} se describen la
metodolog\'ia cl\'asica para el dise\~no de vacunas atenuadas, algunos
antecedentes del dise\~no racional y finalmente, la soluci\'on propuesta. Por
\'ultimo, en la parte~\ref{software} se describen los detalles puramente
t\'ecnicos y mas relevantes sobre el desarrollo del software.

\section{Sobre las vacunas}
\label{vacunas}
Existen diferentes posiciones sobre la efectividad de las vacunas en la
prevenci\'on de las enfermedades. Por un lado, la \textit{versi\'on oficial}
sostiene que representan la principal herramienta para combatir enfermedades y
epidemias. Pero al mismo tiempo, hay quienes aseguran que las vacunas son en
muchos casos, las causantes de las enfermedades que intentan prevenir. Se
cuestionan adem\'as, sus ingredientes t\'oxicos (aluminio, mercurio,
cloroformo), en algunos casos cancer\'igenos o supuestamente relacionados con
diferentes enfermedades como el autismo.

Aun as\'i, su uso esta ampliamente aceptado en la mayor\'ia de los pa\'ises del
mundo siendo, las campa\~nas masivas de vacunaci\'on, una de las principales
pol\'iticas publicas de salud. 

No es el objetivo de este trabajo, profundizar en este tema ni tampoco llegar a
una conclusi\'on apresurada sobre la bondad de las vacunas, sino simplemente
hacer menci\'on a que es un debate abierto. El lector interesado, podr\'a
consultar la bibliograf\'ia tanto a favor como en contra del uso
masivo de vacunas seg\'un lo crea conveniente.

\subsubsection{Un cacho de historia}

El origen de las vacunas se remonta al a\~no 1796 durante la epidemia del virus
de la viruela en Europa. El medico rural, Edward Jenner, observo que las
mujeres que orde\~naban las vacas, eventualmente contra\'ian una especie de
``viruela vacuna'' por el contacto con las ubres y que luego la viruela com\'un
no les produc\'ia ning\'un efecto. Efectivamente, la viruela vacuna es una
variante de la viruela com\'un que no produce efectos de consideraci\'on en las
personas y que dio origen a lo que hoy se conocen como vacunas vivas o
atenuadas.

\section{Motivaci\'on}
\label{motivacion}
Siendo brutalmente honesto, lo que motiv\'o este trabajo fue la necesidad
imperiosa de terminar la Licenciatura. Aunque a medida que me fui
interiorizando en el tema, la problem\'atica que se describe a continuaci\'on lo
podr\'ia haber motivado perfectamente.

Dejando de lado la discusi\'on planteada anteriormente, el objetivo b\'asico de
una vacuna es estimular el sistema inmune sin producir la enfermedad en
cuesti\'on. En este sentido, las vacunas atenuadas presentan algunas ventajas
frente a las denominadas vacunas inactivas.
\begin{itemize}
 \item Proveen inmunidad a largo plazo.
 \item Bajos costos.
 \item Pocas dosis son suficientes para adquirir inmunidad. 
 \item F\'aciles de aplicar.
\end{itemize}

Sin embargo, este tipo de vacunas tambi\'en presentan una importante desventaja,
como es la probabilidad de revertir a la virulencia y que da origen a este
trabajo.

En sucesivas replicaciones, el virus atenuado puede acumular mutaciones en su
secuencia de \ac{RNA} que le devuelvan su car\'acter patog\'enico, produciendo
la enfermedad que se deseaba prevenir. A diferencia de los virus \ac{DNA}, los
virus \ac{RNA} poseen una alta frecuencia de mutaciones estimada en 0.1 a 10
mutaciones por genoma replicado\cite{Vignuzzi08}.

Un caso paradigm\'atico es el de la vacuna Sabin contra la poliomielitis,
tambi\'en conocida como \ac{OPV}. Esta vacuna, desarrollada por Albert Sabin en
1957, es una vacuna atenuada administrada por v\'ia oral para prevenir la
poliomielitis. 

A ra\'iz de una campa\~na impulsada por la \ac{WHO} en 1988 y utilizando la
\ac{OPV}, hacia finales de 2002, se hab\'ia logrado interrumpir la
transimisi\'on end\'emica del poliovirus en 209, de los 216 pa\'ises del
mundo\cite{Aylward04}. 

Sin embargo, la alta inestabilidad gen\'etica de esta vacuna dio lugar a
un nuevo grupo de virus conocidos como \ac{cVDPV} que presentan propiedades
similares a las del poliovirus salvaje, incluyendo
neurovirulencia\footnote{Tendencia o capacidad de un microorganismo de afectar
el sistema nervioso.} y responsables de una nueva enfermedad denominada
\ac{VAPP}. 

Diferentes estudios muestran que \ac{VAPP} ocurre a una tasa de
aproximadamente 1 caso cada 750,000 a 1 millon de ni\~nos que reciben la
primer dosis de \ac{OPV}\cite{Aylward04}. Mas a\'un, en Estados Unidos entre
1980 y 1999, el 95\% de los casos registrados de poliomielitis paralitica,
fueron \ac{VAPP}\cite{DeJesus07}.

En Argentina, el ultimo caso registrado de \ac{VAPP} se produjo en el a\~no
2009, en la provincia de San Luis\cite{msal09}. Esto derivo en un Alerta
Epidemiol\'ogico acompa\~nado de fuertes campa\~nas de vacunaci\'on con la
vacuna \ac{OPV}.

\section{Antecedentes}
\label{antecedentes}
Ante estos datos, se reconoce que uno de los obst\'aculos para erradicar la
poliomielitis, es la \ac{OPV} en si misma\cite{Chumakov08}. Luego, surge la
necesidad de buscar nuevas formas de dise\~nar vacunas atenuadas que est\'en
libres de riesgos, o al menos tengan menor probabilidad de revertir a la
virulencia.

Es a partir de esto y de los avances en la virolog\'ia molecular, que empieza
a tomar fuerza la idea de \textbf{racionalizar el dise\~no de vacunas
atenuadas}, de forma tal de poder controlar y cuantificar la atenuaci\'on de las
vacunas\cite{Lauring10}.

Algunos de los m\'etodos propuestos y que analizaremos con mayor detalle mas
adelante, son la deoptimizaci\'on  de codones\cite{Coleman08} y la fidelidad en
la replicaci\'on del virus atenuado\cite{Vignuzzi08}.

\section{Propuesta}
\label{propuesta}
En este trabajo, realizado con la colaboraci\'on de la
\ac{FuDePAN}\footnote{\url{http://www.fudepan.org.ar}}, se propone la
implementaci\'on de un software, que hemos dado en llamar \ac{vac-o}, para el
dise\~no de secuencias de \ac{RNA} que optimicen las vacunas atenuadas como un
problema de \textbf{``optimizaci\'on combinatoria basado en restricciones''}. 

Este tipo de problemas consiste en asignar valores a un conjunto finito de
variables que satisfagan determinadas condiciones o restricciones. Estas
variables conforman las ``componentes de la soluci\'on'', y las combinaciones de
los distintos valores que puede tomar cada componente forman las potenciales
soluciones del problema. Luego, usando una funci\'on de evaluaci\'on sobre las
soluciones, se debe encontrar una soluci\'on, o varias, que maximicen o
minimicen dicha funci\'on.\cite{Hoos04}.

En nuestro caso, las restricciones ser\'an propiedades sobre partes de la
secuencia de \ac{RNA}, como la conservaci\'on de la secuencia aminoac\'idica y
la estructura secundaria. Las soluciones ser\'an secuencias completas de
\ac{RNA} y la funci\'on de evaluaci\'on sobre estas soluciones, que deseamos
maximizar, estar\'a dada por el n\'umero de mutaciones necesario para alcanzar
una secuencia revertante.

En este sentido, la principal innovaci\'on que presenta este trabajo, es la
utilizaci\'on de diferentes algoritmos para la predicci\'on de estructura
secundaria (directa e inversa), con el fin de determinar secuencias de
\ac{RNA} que conserven parte de la estructura secundaria del virus atenuado y
en consecuencia, mantengan la atenuaci\'on y las propiedades como vacuna.

Para guiar el desarrollo del trabajo se tomo como caso testigo la vacuna
\ac{OPV}. Esto nos permiti\'o establecer una serie de requerimientos b\'asicos,
que esperamos sean de utilidad para otras vacunas.

Desde el punto de vista de la implementaci\'on, se puso especial atenci\'on en
utilizar diferentes principios y patrones de \ac{OOP} con
el fin de lograr un software que sea altamente modular y que permita ser
extendido en el futuro con nuevas funcionalidades.

El c\'odigo fuente fue liberado bajo licencia \ac{GPLv3} y puede ser accedido a
trav\'es del repositorio \ac{SVN}\footnote{\url{http://vac-o.googlecode.com}}