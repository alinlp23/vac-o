\chapter{Bioinform\'atica y Biolog\'ia computacional}
\label{bioinformatica}
\epigraph{I can't be as confident about computer science as I can about biology.
Biology easily has 500 years of exciting problems to work on. It's at that
level.}%
{Donald Knuth}

Por tratarse de una disciplina relativamente nueva, nos permitimos una breve
introducci\'on, definici\'on y repaso de los principales problemas abordados
por la misma profundizando en aquellos problemas que est\'an directamente
relacionados con este trabajo. En particular, la predicci\'on de estructura
secundaria.

\section{Bio*}

La biolog\'ia depende en gran parte de la qu\'imica para poder avanzar y esto
dio lugar a lo que se conoce como \textit{bioqu\'imica}. An\'alogamente, la
necesidad de explicar fen\'omenos biol\'ogicos a nivel at\'omico, dio lugar a la
\textit{biof\'isica}. La \textit{biomatem\'atica} por su parte, se enfoca en el
modelado de procesos biol\'ogicos utilizando t\'ecnicas matem\'aticas que
permitan simular y predecir su comportamiento. La enorme cantidad de datos
recopilados por los bi\'ologos y la necesidad de herramientas para
interpretarlos, dio origen a lo que hoy conocemos como
\textit{bioinform\'atica}.

La bioinform\'atica fue precedida por lo que se llam\'o \textit{biolog\'ia
computacional}. Aunque no hay una definici\'on precisa para ninguno de
los dos t\'erminos, la biolog\'ia computacional se caracteriz\'o por enfocarse
en los aspectos te\'oricos/formales de la Ciencia de la Computaci\'on,
mientras que la bioinform\'atica supo estar m\'as relacionada con el
procesamiento de grandes vol\'umenes de datos, usualmente almacenados en la
Internet. Actualmente, se suele hacer referencia a ambos t\'erminos de manera
indiferente.

\subsubsection{Problemas cl\'asicos}

A continuaci\'on presentamos algunos de los problemas o temas de investigaci\'on
abordados por la bioinform\'atica.

\begin{itemize}
 \item An\'alisis de secuencias de \ac{DNA} o \ac{RNA}.
 \item Dise\~no de secuencias de \ac{DNA} o \ac{RNA}.
 \item Predicci\'on de interacci\'on entre prote\'inas.
 \item An\'alisis de mutaciones en el c\'ancer.
 \item Biolog\'ia evolutiva computacional.
\end{itemize}

\section{Predicci\'on de estructura secundaria}

Uno de los problemas que omitimos mencionar anteriormente y que describiremos
con mayor detalle, es el que se conoce como ``predicci\'on de estructura
secundaria''. 

Este problema consiste en, dada una estructura primaria o secuencia de \ac{RNA},
determinar la estructura secundaria correspondiente. Adem\'as, diferentes
secuencias de \ac{RNA} pueden tener la misma estructura secundaria, por lo que
tambi\'en nos interesa determinar para una estructura secundaria dada, las
secuencias de \ac{RNA} que conservan esa estructura. 

En ingl\'es, se suele denominar a estos dos problemas \textit{folding} e
\textit{inverse folding} respectivamente. A continuaci\'on describimos
brevemente las diferentes aproximaciones a cada uno de ellos.

Por lo mencionado en la secciones~\ref{estructura} y \ref{virus}, predecir la
estructura secundaria de una secuencia de \ac{RNA} y conocer las posibles
secuencias que conservan una estructura secundaria determinada es de gran
importancia en este trabajo.

\subsection{Predicci\'on directa (\textit{folding})}
\label{folding}
Existen esencialmente 2 tipos de algoritmos para determinar o predecir la
estructura secundaria de una secuencia de \ac{RNA}.

\begin{itemize} 
 \item \textbf{Predicci\'on por \ac{mfe}}. Propuesto e implementado por Michael
Zuker en 1981\cite{Zuker81}, utiliza programaci\'on din\'amica para encontrar la
estructura secundaria que minimiza la energ\'ia libre.
 \item \textbf{Predicci\'on comparativa}. Utiliza diferentes m\'etodos para
comparar secuencias y estructuras con el fin de obtener una estructura por
``consenso''\cite{Gardner04}.
\end{itemize}

Si bien la ``predicci\'on comparativa'' presenta un incremento en la fidelidad
de los resultados obtenidos con respecto a la ``predicci\'on por \ac{mfe}''
\cite{Gardner04}, este tipo de algoritmos requieren la existencia de un conjunto
de secuencias relacionadas entre s\'i (hom\'ologas) y esto no siempre es
posible. En particular para este trabajo nos interesa poder realizar
predicciones de la estructura secundaria a partir de una sola secuencia, por lo
que la ``predicci\'on comparativa'' fue descartada.

Entre las implementaciones de la ``predicci\'on por \ac{mfe}'', se destacan
\textbf{RNAfold}\cite{Hofacker94} y \textbf{Mfold}\cite{Zuker81}. Ambas
implementan el algoritmo propuesto por Michael Zuker con complejidad
$\mathcal{O}(N^{3})$ donde $N$ es la longitud de la secuencia, aunque
\textbf{RNAfold} se presenta como una versi\'on mejorada y mas eficiente en la
pr\'actica. Como ya mencionamos en la secci\'on~\ref{estructura}, para lograr
esta complejidad polinomial, es necesario suponer la ausencia de
\textit{pseudo-knots}. De lo contrario, est\'a demostrado que el problema es
NP-completo\cite{Lyngso00}.

A continuaci\'on se puede ver un ejemplo de una predicci\'on realizada con
\textbf{RNAfold}.

\begin{verbatim}
sancho@mulata:~$ RNAfold

Input string (upper or lower case); @ to quit
....,....1....,....2....,....3....,....4....,....5....,....6....,.
AAAGGCAACGGCCAU
length = 15
AAAGGCAACGGCCAU
...(((....)))..
 minimum free energy =  -4.40 kcal/mol
\end{verbatim}

El resultado obtenido es precisamente, la energ\'ia libre y la estructura
secundaria representada con par\'entesis y puntos. Donde los pares de
par\'entesis indican las bases ``apareadas'' o ``unidas'' y los puntos, las
bases libres.

\subsection{Predicci\'on inversa (\textit{inverse folding})}
\label{inverse}
Como ya mencionamos, este problema consiste en determinar las secuencias de
\ac{RNA} que tienen una estructura secundaria determinada y es fundamental
para el dise\~no racional de secuencias de \ac{RNA}.

Las principales implementaciones que abordan este problema, lo plantean como un
\ac{CSP} y utilizan variantes de b\'usqueda local estoc\'astica para
resolverlo. La funci\'on objetivo y que se desea minimizar, es la distancia
estructural\footnote{Existen diferentes m\'etodos y algoritmos para calcular
la distancia entre estructuras cuya descripci\'on exceden este trabajo.}
entre la estructura \ac{mfe} de la secuencia soluci\'on\footnote{En este
contexto, una secuencia soluci\'on es aquella cuya predicci\'on de estructura
secundaria da como resultado la estructura buscada.} y la estructura secundaria
dada. 

Si bien la complejidad de este problema no est\'a determinada, a diferencia de
otros \ac{CSP}, la evaluaci\'on de las posibles soluciones es muy costosa ya que
implica la ``predicci\'on \ac{mfe}'' sobre cada secuencia candidata
($\mathcal{O}(N^{3})$). Luego, se deben utilizar diferentes t\'ecnicas que
tiendan a minimizar el numero de predicciones realizadas sobre la secuencia
completa.

En general, las diferentes implementaciones tienen como par\'ametros de
entrada la estructura secundaria y opcionalmente, una secuencia de \ac{RNA}
incompleta (algunas bases indefinidas, generalmente representadas con la
letra N). Se distinguen dos pasos o etapas principales, que marcan las
diferencias entre una y otra implementaci\'on:
\begin{enumerate}
 \item \textbf{Inicializaci\'on:} determinar una secuencia inicial completando
la secuencia dada como par\'ametro. En el caso de no recibir ninguna secuencia
como par\'ametro, se asume una secuencias con todas las bases indefinidas.
 \item \textbf{B\'usqueda local:} mejorar iterativamente la secuencia inicial
hasta alcanzar una secuencia soluci\'on. Es decir, realizar cambios en la
secuencia que tiendan a minimizar la distancia estructural con la estructura
buscada.
\end{enumerate}

Entre estas implementaciones, se destacan \textbf{RNAinverse}\cite{Hofacker94},
\textbf{INFO-RNA}\cite{Busch07} y \textbf{RNA-SSD}\cite{Andronescu03}. Las
\'ultimas dos, se presentan como mejoras a la primera proponiendo diferentes
formas de generar la secuencia inicial e implementando algoritmos de b\'usqueda
local estoc\'astica mas complejos.

Notar que las tres implementaciones dependen de la generaci\'on de n\'umeros
\textit{pseudo aleatorios} y ya que la ``semilla'' utilizada es variable (salvo
\textbf{RNA-SSD} que permite fijar la semilla), se debe tener en cuenta el no
determinismo. Es decir, en sucesivas ejecuciones del algoritmo y utilizando los
mismos par\'ametros, se podr\'ian obtener distintos resultados. Esto sera clave
a la hora de sistematizar y automatizar el uso de estos algoritmos.

A continuaci\'on, mostramos un ejemplo de una predicci\'on inversa usando
\textbf{RNAinverse} sobre la estructura obtenida anteriormente con
\textbf{RNAfold}.

\begin{verbatim}
sancho@mulata:~$ RNAinverse 

Input structure & start string (lower case letters for const 
positions) @ to quit, and 0 for random start string
....,....1....,....2....,....3....,....4....,....5....,....6....,.
...(((....)))..
aaaNNNNNNNNNNNu
length = 15
aaaUAGUCAGCUACu    3 0
\end{verbatim}

N\'otese que la secuencia obtenida, conserva las bases que ya estaban definidas
en la secuencia incompleta que se dio como par\'ametro y que adem\'as, es
distinta a la secuencia sobre la que se hizo la predicci\'on directa
anteriormente.
