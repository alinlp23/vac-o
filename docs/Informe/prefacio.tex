\chapter*{Prefacio}

Cuando empezamos con este trabajo no estaba muy convencido de su relevancia
para las Ciencias de la Computaci\'on, probablemente por estar enmarcado dentro
de lo que conocemos como ``ciencia aplicada'' y por c\'omo muchas veces se
catalogan estos trabajos como ``m\'as simples'' que los de la ``ciencia pura''.

Durante mi paso por la Universidad, pude participar de diferentes discusiones
sobre el rol de la ciencia y la tecnolog\'ia en funci\'on de un proyecto
pol\'itico nacional y popular. Creo que una de las conclusiones a las que
pudimos llegar con mis compa\~neros de militancia es que el debate no se trata
de ``ciencia pura vs. ciencia aplicada'', sino m\'as bien, ?`ciencia para qu\'e
y para qui\'en?.

En ese sentido creo que es indispensable un trabajo interdisciplinario que
permita generar nuevos conocimientos y aplicarlos para la resoluci\'on de
problemas concretos de la sociedad. Las unidades acad\'emicas aisladas y
encerradas en sus propias disciplinas, no pueden m\'as que tender a quedar
incomunicadas de lo que las rodean.

Por otro lado, creo que los esfuerzos individuales de hacer ciencia
``comprometida'' se diluyen y quedan en la nada si no hay un compromiso del
Estado en priorizar e impulsar las \'areas estrat\'egicas que se consideren
m\'as relevantes para el desarrollo del pa\'is. Este trabajo es probablemente
una prueba de ello.

\begin{quotation}
 \em Ese valor potencial que tiene cualquier descubrimiento cient\'ifico es el
que tendr\'ia un ladrillo arrojado en cualquier lugar del pa\'is, si a alguno se
le ocurriera construir all\'i una casa, por casualidad. Es posible, pero no se
puede organizar una sociedad, ni la ciencia de un pa\'is con ese tipo de
criterio.
\begin{flushright}
(Oscar Varsavsky, 1920 - 1976)
\end{flushright}
\end{quotation}

Si el Estado no impulsa la producci\'on p\'ublica de vacunas y medicamentos,
dif\'icilmente este trabajo signifique un aporte al desarrollo nacional. Por el
contrario, es mucho m\'as probable que sea aprovechado por cualquier otro pa\'is
o empresa farmac\'eutica, a quienes seguramente poco les importan estas l\'ineas
``politizadas''.

Ojal\'a este trabajo, que por el momento y con un poco de suerte, es simplemente
un ``ladrillo arrojado en cualquier lugar del pa\'is'', sirva alg\'un d\'ia para
construir los cimientos de un sistema cient\'ifico menos preocupado por los
\textit{papers} publicados en revistas internacionales y m\'as dedicado al
desarrollo de una Ciencia Nacional, Latinoamericana y Popular.
