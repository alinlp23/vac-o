\chapter{Requerimientos}
\epigraph{If you think it's simple, then you have misunderstood the problem.}%
{Bjarne Stroustrup}

Los requerimientos del software fueron provistos por miembros de \ac{FuDePAN} y
quedaron documentados en la ``Especificaci\'on de Requerimientos de Software''
que se anexa a este trabajo. A continuaci\'on haremos menci\'on a los
requerimientos mas relevantes y que nos permitan introducir los principales
aspectos del software.

\section{Descripci\'on general}

Retomando lo dicho en la secci\'on~\ref{vacunas-propuesta}, podr\'iamos
resumir los requerimientos funcionales de \ac{vac-o} de la siguiente manera:

\begin{itemize}
 \item \textbf{Entrada:} Genoma del virus atenuado, genoma del pat\'ogeno o
revertantes y un conjunto de restricciones sobre partes de la secuencias del
virus atenuado.
 \item \textbf{Objetivo:} Satisfaciendo las restricciones impuestas, maximizar
la cantidad de mutaciones necesarias para que el virus atenuado se convierta en
una secuencia pat\'ogena o revertante.
 \item \textbf{Salida:} Una o varias secuencias, candidatas a ``mejorar'' el
virus atenuado.
\end{itemize}

Desde el punto de vista del dise\~no, el sistema deb\'ia ser altamente modular,
de forma que sea apto para encontrar secuencias gen\'omicas para diferentes
tipos de organismos. En este sentido, se debieron seguir los principios de
dise\~no conocidos por el acr\'onimo \textbf{SOLID}\cite{Martin00}.

\begin{itemize}
  \item \ac{SRP}
  \item \ac{OCP}
  \item \ac{LSP}
  \item \ac{ISP}
  \item \ac{DIP}
\end{itemize}

\section{Extensiones}

Uno de los principales requerimientos con que deb\'ia cumplir el software es
que sea vers\'atil. Es decir, que permita ser configurado y extendido de una
forma simple y sin modificar sus componentes centrales.

En este sentido, se propuso desde un comienzo un software que funcionar\'ia en
base a la informaci\'on provista por una extensi\'on o \textit{plugin}.
B\'asicamente, la responsabilidad de una extensi\'on ser\'ia la de proveer a
\ac{vac-o} la informaci\'on relevante para cada virus atenuado que se desee
optimizar. En particular, el genoma del virus atenuado, el genoma del
pat\'ogeno y el conjunto de restricciones que se deben satisfacer durante la
b\'usqueda. Las extensiones tambi\'en ser\'ian responsables de evaluar las
secuencias candidatas y de determinar la estrategia de b\'usqueda.


\section{Regiones combinatorias}

Como ya mencionamos anteriormente, el problema se puede ver como un problema de
``optimizaci\'on combinatoria basado en restricciones''. Luego, un ingrediente
que no podemos dejar de mencionar son precisamente, las restricciones. 

Como vimos en la secci\'on~\ref{bio-esencial}, sobre una secuencia de \ac{RNA}
se distinguen dos tipos de regiones, \ac{ORF} y \ac{UTR}. Estas regiones de la
secuencia o eventualmente ciertas partes de inter\'es, tienen diferentes
propiedades o funciones biol\'ogicas (la estructura secundaria de \ac{RNA} o
los amino\'acidos que codifican) que permiten considerar a la secuencia de
\ac{RNA} como un virus atenuado (reducen la virulencia del pat\'ogeno
manteni\'endolo viable).

Sobre las diferentes partes de inter\'es (para la atenuaci\'on del virus)
sobre una secuencia de \ac{RNA}, que llamaremos ``regiones combinatorias'', se
aplicar\'an restricciones que tiendan a mantener las propiedades o funciones
biol\'ogicas que garantizan la atenuaci\'on del virus. 

En este contexto nos referimos a las ``variantes'' de una regi\'on combinatoria
como los posibles valores ((sub)secuencias de \ac{RNA}) que satisfacen la
restricci\'on impuesta por la regi\'on combinatoria. Luego, podemos definir a
una regi\'on combinatoria de la siguiente manera.

\begin{definition}
\label{region}
Dada una secuencia de \ac{RNA} $S$ de longitud $n$, una regi\'on combinatoria
sobre $S$ sera una 4-upla $(inicio, fin, tipo, eval)$ tal que:
\begin{itemize}
 \item $0 \le inicio < fin \le n$
 \item $tipo$ determina la restricci\'on sobre la regi\'on y por ende, queda
definido un conjunto $\mathcal{V}$ de posibles variantes a la regi\'on.
 \item $eval: \mathcal{V} \rightarrow (0,1)$ es una funci\'on de evaluaci\'on
local a la regi\'on que determina la bondad de una variante determinada.
\end{itemize}
\end{definition}

Si suponemos $N \ge 1$ y $R_{1}, \dots, R_{N}$ regiones combinatorias.
Entonces tendremos que el n\'umero de posibles secuencias que \ac{vac-o}
``deber\'ia'' evaluar, sera igual al cardinal del producto cartesiano
$\mathcal{V}_{1} \times \dots \times \mathcal{V}_{N}$ con  $\mathcal{V}_{1},
\dots, \mathcal{V}_{N}$ variantes de cada regi\'on combinatoria respectivamente.

Con el objetivo de que las secuencias evaluadas cumplan con un m\'inimo criterio
de ``calidad'' y al mismo tiempo restringir el n\'umero de posibilidades, solo
ser\'an tenidas en cuenta aquellas combinaciones de las variantes a cada
regi\'on combinatoria tal que el producto de sus respectivas evaluaciones
locales ($eval$) sea mayor o igual a un ``umbral'' determinado por la
extensi\'on. Es decir, ser\'an tenidas en cuenta las combinaciones de variantes:
\begin{center}
$(v_{1},\dots, v_{N}) \in \mathcal{V}_{1} \times \dots \times \mathcal{V}_{N}$
tal que $\prod_{i=1}^{N} eval(v_{i}) \ge umbral$ 
\end{center}



Para el caso testigo de la \ac{OPV}, se contemplaron dos tipos de regiones
combinatorias aunque se debe tener en cuenta la posibilidad de que en el futuro
se agreguen mas. Estos tipos de regiones combinatorias o restricciones son:
\begin{itemize}
 \item Conservaci\'on de la estructura secundaria de \ac{RNA}.
 \item Conservaci\'on del c\'odigo gen\'etico.
\end{itemize}

Adem\'as, para el caso de la ``conservaci\'on de estructura secundaria de
\ac{RNA}'' se deb\'ia contar con la posibilidad de utilizar diferentes
algoritmos de predicci\'on directa e inversa de manera ``transparente''.

\section{Estrategias de b\'usqueda}

Otro de los aspectos ``configurables'' del software deb\'ia ser la estrategia
de b\'usqueda utilizada. En este sentido, la extensi\'on es responsable de
proveer al sistema la ``forma'' de recorrer el espacio de b\'usqueda y al mismo
tiempo determinar cuando se debe terminar el recorrido.

Vale aclarar que todos los algoritmos de b\'usqueda local son ``incompletos''
en el sentido de que no recorren exhaustivamente el espacio de b\'usqueda como
si lo hacen los algoritmos mas tradicionales o sistem\'aticos. Luego, es
necesario contar con un criterio de terminaci\'on que suele estar relacionado
con la cantidad de iteraciones realizadas, la ``calidad'' de las soluciones
encontradas o la cantidad de iteraciones sin que se produzcan mejores
soluciones.

\section{Control de calidad}

Cada secuencia encontrada por \ac{vac-o} durante el recorrido del espacio
de b\'usqueda, debe ser sometida a un ``control de calidad'' complementario al
requerimiento sobre la calidad de las variantes a cada regi\'on combinatoria
que mencionamos anteriormente ($\prod_{i=1}^{N} eval(v_{i}) \ge umbral$).

B\'asicamente se pretende simular las mutaciones acumuladas que se producen en
las sucesivas replicaciones del virus en la naturaleza y sobre cada secuencia
mutante, verificar determinadas propiedades que brinden mayor seguridad sobre la
atenuaci\'on del virus.

An\'alogamente a las regiones combinatorias, definimos las regiones de
validaci\'on de la siguiente manera.

\begin{definition}
Dada una secuencia de \ac{RNA} $S$ de longitud $n$, una regi\'on de validaci\'on
sobre $S$ sera una 5-upla $(inicio, fin, prueba, criterio, prof)$ tal que:
\begin{itemize}
 \item $0 \le inicio < fin \le n$
 \item $prueba$ determina la forma en que se producen las mutaciones.
 \item $criterio$ determina la propiedad que deben cumplir todas las secuencias
mutantes generadas.
 \item $prof>0$ determina la profundidad con que se realiza el control de
validaci\'on (ciclos de replicaci\'on que se simulan).
\end{itemize}
\end{definition}

Se puede pensar al control de calidad como la generaci\'on de un \'arbol de
secuencias. La ra\'iz del \'arbol ser\'a la secuencia candidata y los nodos del
\'arbol ser\'an las secuencias mutantes generadas mediante la forma seleccionada
en la variable $prueba$ y que cumplan con la propiedad determinada por la
variable $criterio$. Si una secuencia mutante no cumple con esa propiedad, no se
genera ese nodo del \'arbol, se corta en ese punto el control de calidad y la
secuencia candidata no pasa el control de calidad sobre esa regi\'on de
validaci\'on.

Diremos que una secuencia pasa el control de calidad sobre una regi\'on de
validaci\'on, cuando sea posible generar todos los nodos del \'arbol hasta
alcanzar la profundidad $prof$. Luego, solo ser\'an consideradas aquellas
secuencias candidatas que pasen el control de calidad sobre todas las regiones
de validaci\'on.

Para las formas de producir mutaciones acumuladas se contemplaron, ``mutaciones
sistem\'aticas'' y ``mutaciones aleatorias''. Por otro lado, las propiedades
que se contemplaron como criterio de verificaci\'on son, ``similitud
estructural'' y ``disimilitud estructural''.

Por similitud y disimilitud estructural nos referimos a la comparaci\'on entre
la estructura secundaria de \ac{RNA} de cada secuencia mutante y una estructura
secundaria de \ac{RNA} dada por la extensi\'on. 

Si suponemos $\mathcal{E}$ el conjunto de estructuras secundarias sobre
secuencias de longitud $n$ y una funci\'on $StructureCompare : \mathcal{E}
\times \mathcal{E} \rightarrow (-\infty, 1)$ tal que dadas $e_{1},e_{2} \in
\mathcal{E}$ calcula la similitud entre ellas, siendo 1 el valor de similitud
m\'axima ($e_{1} = e_{2}$). Entonces, para $s \in (-\infty, 1)$ tenemos que:
\begin{itemize}
 \item Similitud estructural: $StructureCompare(e_{1}, e_{2}) \ge s$
 \item Disimilitud estructural: $StructureCompare(e_{1}, e_{2}) \le s$
\end{itemize}

De todas formas, se deb\'ia contemplar la posibilidad de que en el
futuro se agreguen otras formas de generar mutaciones y otros criterios de
verificaci\'on.

\section{Ranking de secuencias candidatas}

Finalmente, nos queda por describir el ranking de secuencias candidatas y que
representa fundamentalmente el resultado de la ejecuci\'on de \ac{vac-o}.

Esto es simplemente un listado de secuencias de \ac{RNA}, ordenadas seg\'un la
evaluaci\'on que hiciera la extensi\'on. B\'asicamente, las secuencias mejor
evaluadas ser\'an aquellas que est\'en a mayor distancia (para alguna
definici\'on de distancia) de la secuencia pat\'ogena o revertante y por ende,
tendr\'an menor probabilidad de sufrir reversi\'on a la virulencia.